\documentclass[10pt,a4paper,ngerman]{article}
\usepackage[margin=2.5cm]{geometry}
\usepackage[T1]{fontenc}
\usepackage[utf8]{inputenc} % Zeichensatz
\usepackage[ngerman]{babel} % Sprachpaket
\usepackage{amsmath}  % Mathematik
\usepackage{amsfonts} % Mathematik
\usepackage{amssymb}  % Mathematik
\usepackage{palatino} % Schriftart
\usepackage{titling}  % für eigene Überschrift
\usepackage{graphicx}
\usepackage{wasysym}  % enthält Symbole, wie Quadrate (für Multiple Choice Fragen)
\usepackage{dirtree}  % Verzeichnisbäume
\usepackage{hyperref} % Hyperlinks und andere Verlinkungen
\usepackage{enumitem} % Listen modifizieren (Abtände usw.)
\usepackage{rotating} % Um den Text zu rotieren

% Package für die Kopf- und Fußzeilen
\usepackage{scrlayer-scrpage}
\pagestyle{scrheadings}
\clearpairofpagestyles % Die einzelnen Bereiche leeren

% Quelltext-Listings 
\usepackage{listings} % Listings
\usepackage{color}    % Syntax-Highlighting

% Farben definieren (für Syntax-Highlighting)
\definecolor{middlegray}{rgb}{0.5,0.5,0.5}
\definecolor{lightgray}{rgb}{0.8,0.8,0.8}
\definecolor{darkgray}{gray}{0.2} % gray: nur ein Wert wird angegeben, welcher dem Grauton entspricht
\definecolor{comment}{rgb}{0.0,0.5,0.0}
\definecolor{keywordcolor}{rgb}{0.0, 0.28, 0.67}
\definecolor{stringcolor}{rgb}{0.75, 0.24, 0.16}

\newcommand{\lstfs}{\fontsize{10}{12}}

% Listings formatieren
\lstset{
	language=python,
   	basicstyle=\ttfamily\lstfs,
   	keywordstyle=\bfseries\ttfamily\color{keywordcolor},
   	stringstyle=\color{stringcolor}\ttfamily,
   	commentstyle=\color{comment}\ttfamily,
  	emph={square}, 
   	emphstyle=\ttfamily,
   	emph={[2]root,base},
   	emphstyle={[2]\ttfamily},
   	showstringspaces=false,
   	flexiblecolumns=false,
   	tabsize=2,
   	numbers=left,
   	numberstyle=\tiny,
   	numberblanklines=false,
   	stepnumber=5,
   	firstnumber=1,
 	numberfirstline=true,
   	numbersep=10pt,
	xleftmargin=15pt,
	literate={ö}{{\"o}}1
           {ä}{{\"a}}1
           {ü}{{\"u}}1
           {ß}{{\ss}}1
}


% ÜBERSCHRIFTEN
\usepackage{titlesec}

\titleformat{\section}
{\normalfont\Large\bfseries Aufgabe}
{ \thesection : }{0em}{}

\author{Python-Übungen -- Tim Poerschke (\href{http://timkodiert.de}{timkodiert.de})}


% Metainformationen
\title{Kontrollstrukturen}

\ihead{}
\chead{}
\ohead{}
\ifoot{}
\cfoot{\pagemark}
\ofoot{}


\newcommand{\doctitle}[1]{\begin{center}\begin{huge}#1\end{huge}\end{center}}

% 1. Variante des docheader (ohne Parameter)
% dann wird der Autor unter der Überschrift ausgegeben
\newcommand{\docheader}{\doctitle{\thetitle}\begin{center}\theauthor\end{center}\hrule\vspace{1em}}

% 2. Variante des docheader (mit Parameter)
% was im Parameter steht wird unter der Überschrift ausgeben
\newcommand{\docheaderparam}[1]{\doctitle{\thetitle}\begin{center}#1\end{center}\hrule\vspace{1em}}

\newcommand{\timbox}[1]{\begin{center}\fbox{\begin{minipage}[t]{0.8\textwidth}#1\end{minipage}}\end{center}}

\newcommand{\code}[1]{\texttt{#1}}

% Für Multiple Choice
\newcommand{\choice}{\item[\Square]}
\newcommand{\cchoice}{\item[\CheckedBox]} % cc = correc choice

% Hinweis, wo die Lösung zu finden ist.
\newcommand{\answerlocation}[1]{Die Lösung finden Sie in der Datei \code{\thesection\_#1.py}.}

\setlength{\parskip}{1em}
\setlength{\parindent}{0em}
\renewcommand{\baselinestretch}{1.15}

\begin{document}

\docheader

\section{Ping Pong! \small{(if-Verzweigungen)}}

Schreiben Sie ein Programm, dass auf eine Nutzereingabe reagiert. Das Programm soll "`Pong"' ausgeben, wenn der Nutzer "`Ping"' eingegeben hat. Bei allen anderen Eingaben soll "`Falsche Eingabe"' angezeigt werden.

\textit{Hinweis: Die Nutzereingabe können Sie wie folgt abfragen.}
\vspace{1em}
\begin{lstlisting}
user_input = input("Eingabe bitte: ")
\end{lstlisting}

\answerlocation{PingPong}


\section{Eingaben vergleichen \small{(if-Verzweigungen)}}

Fordern Sie zwei Nutzereingaben an und konvertieren Sie diese jeweils in einen Integer. Überprüfen Sie, welche der Eingaben die größere Zahl war. Mögliche Ausgaben des Programms:

\code{Eingabe 1 war größer} \\
\code{Eingabe 2 war größer} \\
\code{Die Eingaben waren identisch}

\answerlocation{EingabenVergleichen}


\section{Was ist die Ausgabe? \small{(if-Verzweigungen)}}

\lstinputlisting{../../02_Kontrollstrukturen/3_AusgabeIf.py}

Was wird dieses Programm ausgeben? Versuchen Sie die Ausgabe herauszufinden, ohne das Skript auszuführen. Um Ihr Ergebnis zu prüfen, können Sie das Programm \code{3\_AusgabeIf.py} ausführen.


\section{Summieren wie die Sumerer \small{(Schleifen)}}

Implementieren Sie ein Python-Programm, welches die folgende Summenformel berechnet:

$$\sum_{x=3}^{13}(x-2)^2$$

Passen Sie das Programm anschließend so an, dass der Nutzer die Unter- und Obergrenze (jeweils inklusiv) angeben kann. Dazu können Sie die \code{input}-Funktion oder die Kommandozeilenparameter verwenden. 

\textit{Optional:} Wandeln Sie die \code{while}-Schleife in eine äquivalente \code{for}-Schleife um.

\answerlocation{Summieren}


\section{Pattern Printing \#1 \small{(Schleifen)}}

Entwerfen Sie ein Python-Programm, welches das nachfolgende Muster ausgibt: 

\code{1 22 333 4444 5555 ...}

\answerlocation{PatternPrinting1}


\section{Pattern Printing \#2 \small{(Schleifen)}}

Entwerfen Sie ein Python-Programm, welches das nachfolgende Muster ausgibt: 

\begin{lstlisting}
    *
   * *
  *   *
 *     *
*       *
\end{lstlisting}

\answerlocation{PatternPrinting2}

\end{document}
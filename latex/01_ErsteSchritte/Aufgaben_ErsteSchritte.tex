\documentclass[10pt,a4paper,ngerman]{article}
\usepackage[margin=2.5cm]{geometry}
\usepackage[T1]{fontenc}
\usepackage[utf8]{inputenc} % Zeichensatz
\usepackage[ngerman]{babel} % Sprachpaket
\usepackage{amsmath}  % Mathematik
\usepackage{amsfonts} % Mathematik
\usepackage{amssymb}  % Mathematik
\usepackage{palatino} % Schriftart
\usepackage{titling}  % für eigene Überschrift
\usepackage{graphicx}
\usepackage{wasysym}  % enthält Symbole, wie Quadrate (für Multiple Choice Fragen)
\usepackage{dirtree}  % Verzeichnisbäume
\usepackage{hyperref} % Hyperlinks und andere Verlinkungen
\usepackage{enumitem} % Listen modifizieren (Abtände usw.)
\usepackage{rotating} % Um den Text zu rotieren

% Package für die Kopf- und Fußzeilen
\usepackage{scrlayer-scrpage}
\pagestyle{scrheadings}
\clearpairofpagestyles % Die einzelnen Bereiche leeren

% Quelltext-Listings 
\usepackage{listings} % Listings
\usepackage{color}    % Syntax-Highlighting

% Farben definieren (für Syntax-Highlighting)
\definecolor{middlegray}{rgb}{0.5,0.5,0.5}
\definecolor{lightgray}{rgb}{0.8,0.8,0.8}
\definecolor{darkgray}{gray}{0.2} % gray: nur ein Wert wird angegeben, welcher dem Grauton entspricht
\definecolor{comment}{rgb}{0.0,0.5,0.0}
\definecolor{keywordcolor}{rgb}{0.0, 0.28, 0.67}
\definecolor{stringcolor}{rgb}{0.75, 0.24, 0.16}

\newcommand{\lstfs}{\fontsize{10}{12}}

% Listings formatieren
\lstset{
	language=python,
   	basicstyle=\ttfamily\lstfs,
   	keywordstyle=\bfseries\ttfamily\color{keywordcolor},
   	stringstyle=\color{stringcolor}\ttfamily,
   	commentstyle=\color{comment}\ttfamily,
  	emph={square}, 
   	emphstyle=\ttfamily,
   	emph={[2]root,base},
   	emphstyle={[2]\ttfamily},
   	showstringspaces=false,
   	flexiblecolumns=false,
   	tabsize=2,
   	numbers=left,
   	numberstyle=\tiny,
   	numberblanklines=false,
   	stepnumber=5,
   	firstnumber=1,
 	numberfirstline=true,
   	numbersep=10pt,
	xleftmargin=15pt,
	literate={ö}{{\"o}}1
           {ä}{{\"a}}1
           {ü}{{\"u}}1
           {ß}{{\ss}}1
}


% ÜBERSCHRIFTEN
\usepackage{titlesec}

\titleformat{\section}
{\normalfont\Large\bfseries Aufgabe}
{ \thesection : }{0em}{}

\author{Python-Übungen -- Tim Poerschke (\href{http://timkodiert.de}{timkodiert.de})}


% Metainformationen
\title{Erste Schritte}

\ihead{}
\chead{}
\ohead{}
\ifoot{}
\cfoot{\pagemark}
\ofoot{}


\newcommand{\doctitle}[1]{\begin{center}\begin{huge}#1\end{huge}\end{center}}

% 1. Variante des docheader (ohne Parameter)
% dann wird der Autor unter der Überschrift ausgegeben
\newcommand{\docheader}{\doctitle{\thetitle}\begin{center}\theauthor\end{center}\hrule\vspace{1em}}

% 2. Variante des docheader (mit Parameter)
% was im Parameter steht wird unter der Überschrift ausgeben
\newcommand{\docheaderparam}[1]{\doctitle{\thetitle}\begin{center}#1\end{center}\hrule\vspace{1em}}

\newcommand{\timbox}[1]{\begin{center}\fbox{\begin{minipage}[t]{0.8\textwidth}#1\end{minipage}}\end{center}}

\newcommand{\code}[1]{\texttt{#1}}

% Für Multiple Choice
\newcommand{\choice}{\item[\Square]}
\newcommand{\cchoice}{\item[\CheckedBox]} % cc = correc choice

% Hinweis, wo die Lösung zu finden ist.
\newcommand{\answerlocation}[1]{Die Lösung finden Sie in der Datei \code{\thesection\_#1.py}.}

\setlength{\parskip}{1em}
\setlength{\parindent}{0em}
\renewcommand{\baselinestretch}{1.15}

\begin{document}

\docheader

\section{Variablen}

Schreiben Sie ein Python-Programm, welches vier Variablen enthält: 

\vspace{1em}
\renewcommand{\arraystretch}{1.5}
\begin{tabular}{|l|l|}
\hline
Variable & Wert \\
\hline
a & \code{"Hallo Welt!"} \\[-0.5em]
b & \code{42} \\[-0.5em]
c & \code{3.14} \\[-0.5em]
d & \code{'x'} \\
\hline
\end{tabular}
\vspace{1em}

Das Programm soll die Variablen ausgeben. Und: Welche Datentypen besitzen die Variablen?

Die Lösung finden Sie in der Datei \code{1\_Variablen.py}.


\section{Operatoren}

Schreiben Sie ein weiteres Programm. Das Programm soll wieder vier Variablen enthalten. Die erste beinhaltet "`Ottostraße"', die zweite "`29"' (als Integer), die dritte "`44867"' (als Integer) und die letzte "Bochum". Konkatenieren Sie die Variablen so, dass Sie folgende Ausgabe in der Konsole erhalten: \code{Ottostraße 29, 44867 Bochum}

Definieren Sie nun eine weitere Variablen mit den Werten 128 (\code{x}) und 16 (\code{y}). Inkrementieren Sie den Wert von y um 1. Führen Sie nun eine ganzzahlige Division durch, wobei \code{x} der Dividend und \code{y} der Divisor ist. Speichern Sie das Ergebnis in einer Variable. Schreiben Sie auch den Divisionsrest in eine Variable. Geben Sie die beiden Ergebnisse in der Konsole aus. 

Die Lösung finden Sie in der Datei \code{2\_Operatoren.py}.

\section{Was ist die Ausgabe?}

\lstinputlisting{../../01_ErsteSchritte/3_Ausgabe.py}

Was wird dieses Programm ausgeben? Versuchen Sie die Ausgabe herauszufinden, ohne das Skript auszuführen. Um Ihr Ergebnis zu prüfen, können Sie das Programm mit dem Befehl \code{python 3\_Ausgabe.py} ausführen.

\end{document}
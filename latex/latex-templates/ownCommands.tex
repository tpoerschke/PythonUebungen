
\newcommand{\doctitle}[1]{\begin{center}\begin{huge}#1\end{huge}\end{center}}

% 1. Variante des docheader (ohne Parameter)
% dann wird der Autor unter der Überschrift ausgegeben
\newcommand{\docheader}{\doctitle{\thetitle}\begin{center}\theauthor\end{center}\hrule\vspace{1em}}

% 2. Variante des docheader (mit Parameter)
% was im Parameter steht wird unter der Überschrift ausgeben
\newcommand{\docheaderparam}[1]{\doctitle{\thetitle}\begin{center}#1\end{center}\hrule\vspace{1em}}

\newcommand{\timbox}[1]{\begin{center}\fbox{\begin{minipage}[t]{0.8\textwidth}#1\end{minipage}}\end{center}}

\newcommand{\code}[1]{\texttt{#1}}

% Für Multiple Choice
\newcommand{\choice}{\item[\Square]}
\newcommand{\cchoice}{\item[\CheckedBox]} % cc = correc choice